\documentclass{report}

\usepackage{tabularx}
\usepackage{graphicx}
\usepackage{alltt}

\addtolength{\oddsidemargin}{-0.25in}
\addtolength{\textwidth}{1in}
\addtolength{\textheight}{1.5in}
\addtolength{\voffset}{-0.5in}
\linespread{1.3}
\setlength{\parindent}{0in}
\setlength{\parskip}{1.5ex plus 0.5ex minus 0.2ex}


\newcommand{\tablewidth}{\textwidth}
\newcommand{\mathdialogue}[2]
{
  \begin{center}
  \begin{tabular}[t]{rl}
  {\tt In := } & \parbox{10cm}{\tt {#1}} \\
\\
  {\tt Out = } & \parbox{10cm}{#2} \\
  \end{tabular}
  \end{center}
}
\newcommand{\mathinput}[1]
{
  \begin{tt}
  \begin{center}
    #1
  \end{center}
  \end{tt}
}
\newcommand{\Tud}[3]{  #1 ^#2 _{\phantom{#2} #3}}

\newcommand{\fixme}[1]{\textcolor{red}{#1}}

\title{Kranc User Guide}
\author{Sascha Husa and Ian Hinder}


\begin{document}
\maketitle
\tableofcontents
\chapter{Introduction}

\section{Kranc}
Kranc is a suite of Mathematica-based computer-algebra packages, which
comprise a toolbox to convert certain (tensorial) systems of partial
differential evolution equations to parallelized C code for
solving initial boundary value problems.  Kranc can be used as a rapid
prototyping system for physicists or mathematicians handling
complicated systems of partial differential equations, but through
integration into the Cactus computational toolkit it is also possible
to produce efficient parallelized production codes.  Our work is
motivated by the field of numerical relativity, where Kranc is used as
a research tool by the authors.  The initial version of Kranc was
described in \cite{KrancPaper}, and subsequent enhancements in
\cite{IHPhDThesis}.  The user-visible portion of Kranc has
subsequently been redesigned.  The material in this document is drawn
from these two sources, and has been updated to be consistent with the
current version of Kranc.

\section{Cactus}

The {\em Cactus Computational Toolkit} is an open-source problem
solving environment originally developed in the numerical relativity
community.  It is arranged as a central {\em flesh} and a collection
of modules called {\em thorns} which all communicate with the
flesh.  Many thorns are provided, and the user writes additional thorns
in C or Fortran which solve their particular physics problem.  Cactus
is particularly suited to the numerical solution of time dependent
partial differential equations.

Kranc is concerned with taking an abstract mathematical description of
a system of PDEs and producing working computer code.  It does this by
generating Cactus thorns, allowing use of all the infrastructure
provided by Cactus.

For example, Kranc makes uses of existing Cactus thorns which provide:
\begin{itemize}
\item Parameter file parsing.
\item Memory management for variables associated with the
computational grid.
\item Scheduling of parts of the code based upon parameters.
\item Standard efficient time integrators such as fourth order
Runge-Kutta and iterative Crank-Nicolson via the {\em MoL} thorn
written by I.~Hawke.
\item Mesh refinement \cite{Schnetter}; i.e., using variable resolution
across the numerical grid, so that the computational resources are
focused on interesting parts of the simulation.
\item Automatic parallelization of the code to run across multiple
processors on a supercomputer or cluster, both to improve
computational speed and to use larger grids than can be stored in the
memory of a single node.
\item Output of grid variables to permanent storage in a structured
format.
\end{itemize}
These tasks are completely separate from the physics and numerical
analysis side of the problem, but are necessary in most numerical
codes.

\section{Overview of the Kranc system}

Kranc provides a Mathematica function called {\em CreateThorn}.  The
user must construct arguments and data structures for this function
describing the thorn they wish to create.  Kranc generates code and
Cactus CCL files for:

\begin{itemize}
\item Declaring the grid functions which the simulation will use;
\item Registering the grid functions for the evolved variables with
  the MoL thorn;
\item Computing the right hand sides of evolution equations so that the
time integrator can compute the evolved variables at the next time
step;
\item Computing finite differences, both using built-in definitions of
  standard centred finite differencing operators, as well as allowing
  the user to create customised operators;
\item Performing a user-specified calculation at each point of the grid.
\end{itemize}

User-specified calculations will typically set certain grid variables
as functions of others, and can be used for various purposes including
making a change of variables or computing intermediate or analysis
quantities from evolved variables.

The most important data structure in Kranc is a {\em Calculation}
structure.  It encapsulates the idea of assigning new values to grid
functions in a loop over grid points based upon evaluating expressions
involving other grid functions.  Calculations contain lists of
assignment statements for different grid functions, and these are
evaluated at each point on the grid.  Calculations can also contain
temporary variables called {\em shorthands} into which are placed
intermediate expressions which are used later in the
calculation.  Calculations also contain additional information needed
by the Kranc system, such as a name for the calculation.

\chapter{Using Kranc}

%% \section{Types of arguments}

%% Mathematica allows two types of arguments to be passed to a function.
%% {\em positional arguments} and {\em named arguments} (referred to in
%% the Mathematica book as {\em optional arguments}).  It is possible for
%% some named arguments to be omitted from a function call; in this case
%% a suitable default will be chosen.  Positional arguments are useful
%% when there are few arguments to a function, and their meaning is clear
%% in the calling context.  Named arguments are preferred when there are
%% many arguments, as the argument names are given explicitly in the
%% calling context.  Named arguments are given after the positional
%% arguments in the form \verb|ArgumentName -> argumentvalue|.  For
%% example:

%% \begin{center}
%% \begin{minipage}{0.8 \textwidth}
%% \begin{verbatim}
%% f[x, y, Sum -> True, Verbose -> True]
%% \end{verbatim}
%% \end{minipage}
%% \end{center}
%% Here the x and y are positional arguments, and Sum and Verbose are
%% named arguments.

\section{Data structures}

The user needs to pass information to Kranc in a structured way.
Mathematica does not have the concept of a C++ class or a C structure,
in which collections of named objects are grouped together for ease of
manipulation.  Instead, we have defined a {\em Kranc structure} as a
list of rules of the form {\tt {\it key} -> {\it value}}.  We have
chosen to use the Mathematica rule symbol ``{\tt ->}'' for syntactic
convenience.

For example, one might describe a person using a ``Person'' structure
as follows:

\begin{center}
\begin{minipage}{0.8 \textwidth}
\begin{verbatim}
alice = {Name -> "Alice",
         Age -> 20,
         Gender -> Female}
\end{verbatim}
\end{minipage}
\end{center}

Based on this concept a number of data structures have been defined
which will be used to describe the thorn to construct.  Each of these
data structures is introduced below.


\subsection{Data structure: Calculation}

Calculation structures are the core of the Kranc system.  A Calculation
structure has the following form:

\begin{center}
\begin{tabularx}{\tablewidth}{|l|X|X|l|}
  \hline
  \bf Key & \bf Type & \bf Description & \bf Default\\
  \hline
  Name & String & The name of the calculation & (none) \\

  Equations & List of rules & The assignments that this calculation will perform & \{\} \\
  Schedule & List of Strings & Cactus schedule specifications & Automatic \\
  Shorthands & List of Symbols & Temporary variables which will be used in this calculation & \{\} \\
  CollectList & List of Symbols & Variables which will be used by Collect in Simplify & \{\} \\
  Where &  Everywhere / Interior / Boundary & Which part of the grid this calculation will be performed on & Everywhere \\
  NoSimplify & True/False & Whether to disable simplification of the equations for this calculation & False \\
  ConditionalOnTextuals & List of Strings & Conditional expressions to be inserted in the Cactus schedule.ccl & \{\} \\
  DeclarationIncludes & List of Strings & Cactus include files to use in this thorn & \{\} \\
  \hline
\end{tabularx}
\end{center}

Only the {\it Name} key is required; all the others take default
values if omitted.

\subsubsection{Name}

The name of a calculation is a string which will be used as the
function name in Cactus, as well as the base of the filename of the
source file implementing the calculation in the generated thorn.  For
example,

\begin{center}
\begin{minipage}{0.8 \textwidth}
\begin{verbatim}
Name -> "wave_calc_rhs"
\end{verbatim}
\end{minipage}
\end{center}

\subsubsection{Schedule}

This is a Cactus schedule specification describing
when in the simulation the calculation will be performed.  At its
simplest, it can be given as one of the following:
\begin{itemize}
\item {\tt "at INITIAL"}
\item {\tt "in MoL\_CalcRHS"}
\item {\tt "at ANALYSIS"}
\end{itemize}
so that the calculation is performed in either the initial data time
bin, the schedule group for calculating right hand sides of evolution
equations, or the analysis time bin.  This is given as a list to allow
the calculation to be scheduled at multiple points.  Omitting the
schedule information causes Kranc to schedule the calculation
automatically.  Currently, this is used for analysis quantities which
are scheduled in \verb|MoL_PseudoEvolution| and have boundary
conditions applied after them.

\subsubsection{Shorthands}

This is a list of variables which are to be considered as
shorthands for the purposes of this calculation.  These are variables
which are defined locally in the loop and are not grid functions
defined over the whole grid.  They are used as temporary intermediate
variables when setting more complicated grid functions.  For example,
\begin{center}
\begin{minipage}{0.8 \textwidth}
\begin{verbatim}
Shorthands -> {a,b}
\end{verbatim}
\end{minipage}
\end{center}
would define two shorthands called $a$ and $b$ which can be assigned
to and used in the equations.

\subsubsection{Equations}

The equation list is a list of assignments to perform in the
calculation loop.  Each equation is of the form
\begin{center}
\begin{minipage}{0.8 \textwidth}
\begin{verbatim}
variable -> expression
\end{verbatim}
\end{minipage}
\end{center}
When the calculation is performed, for each point in the grid, {\it
expression} is evaluated and placed into the grid function {\it
variable}.  Here {\it expression} may contain partial derivatives of
grid functions which have been defined in a PartialDerivatives
structure.  {\it variable} may be either a grid function or a
shorthand.  Using the notation {\tt dot[{\it gf}]} for {\it variable}
represents a time derivative of the grid function {\it gf}; this
should be used when the calculation is scheduled in {\tt MoL\_CalcRHS}
for calculating the right hand sides of the evolution equations.

The arrangement of the terms in the equations can have a marked effect
on both compile time and run time.  It is often helpful to tell
Mathematica to collect the coefficients of certain types of term,
rather than expanding out entire expressions.  To this end, the user
can include a {\em CollectList} entry in a calculation; this is a list
of variables whose coefficients should be collected during
simplification.

Note that the system is not designed to allow the same grid function
to be set more than once in a single loop of a calculation.  This
limitation may be removed in the future.

\subsubsection{Example}

The following example is taken from the Kranc implementation of the
NOR formulation of Einstein's equations.  It is a calculation which
describes the time evolution equation for the lapse function $\alpha$
in {\em harmonic slicing}.  It uses the TensorTools package to
represent tensorial quantities.

\begin{center}
\begin{minipage}{0.8 \textwidth}
\begin{verbatim}
lapseEvolveCalc = 
{
  Name -> "nor_harmonic_slicing",
  Schedule -> {"in MoL_CalcRHS"},
  Shorthands -> {trK, hInv[ua,ub]},
  Equations -> 
  {
     hInv[ua,ub] -> MatrixInverse[h[ua,ub]],
     trK -> K[la,lb] hInv[ua,ub],
     dot[alpha] -> alpha^2 trK
  }
};
\end{verbatim}
\end{minipage}
\end{center}
(The {\it MatrixInverse} function is provided by TensorTools to
generate an expression for the inverse matrix.)



\subsection{Data structure: PartialDerivatives}

The user can define partial derivative operators and associated finite
difference approximations of these operators.  This allows different
discretizations of the PDE system.

A finite difference operator maps grid functions to grid functions.
We restrict to those operators which are polynomials in {\em shift}
operators.  In one dimension, the shift operator $E_+$ is defined as 
\begin{eqnarray}
E_+ v_j \equiv v_{j+1}
\end{eqnarray}
It is clear that
\begin{eqnarray}
(E_+)^n v_j = v_{j+n}
\end{eqnarray}
and negative powers $n$ take on the obvious meaning.  In three
dimensions, there is one shift operator for each dimension:
\begin{eqnarray}
E_{+1} v_j \equiv v_{j+(100)} \qquad
E_{+2} v_j \equiv v_{j+(010)} \qquad
E_{+3} v_j \equiv v_{j+(001)}
\end{eqnarray}
where here $j = j_1 j_2 j_3$ is a multi-index.

The PartialDerivatives structure is a list of definitions of partial
derivative operators in terms of finite difference approximations:

\begin{center}
\begin{minipage}{0.8 \textwidth}
\begin{tt}
\{  {\it name}[i\_, j\_, \ldots] -> {\it defn}, ...  \}
\end{tt}
\end{minipage}
\end{center}

where {\it name} is the name for the partial derivative, and {\it
defn} is an algebraic expression in shift operators representing the
difference operator.  The shift operator $E_{+i}$ is written as {\tt
shift[i]}.  The form {\tt spacing[i]} can be used in {\it defn} to
represent the grid spacing in the $i$ direction.  The parameters {\tt
i, j, \ldots} are used in {\it defn} to represent the direction of
differentiation for the first, second, etc.  derivatives.  Partial
derivatives with the same name but a different number of arguments
(i.e., for first and second derivatives) are allowed in the
PartialDerivatives structure.

Since the definitions of the difference operators are written in terms
of Mathematica expressions, higher level operators can be constructed
from {\tt shift} and {\tt spacing}.  For example, Kranc predefines

\begin{center}
\begin{minipage}{0.8 \textwidth}
\begin{verbatim}
DPlus[n_]  := (shift[n] - 1)/spacing[n];
DMinus[n_] := (1 - 1/shift[n])/spacing[n];
DZero[n_]  := (DPlus[n] + DMinus[n])/2;
\end{verbatim}
\end{minipage}
\end{center}

As an example, we give here a PartialDerivatives structure containing
the definition of the standard second order accurate difference
operators, as well as the $D_0^2$ discretization.

\begin{center}
\begin{minipage}{0.8 \textwidth}
\begin{verbatim}
derivs = {
  PDstandard2nd[i_] -> DZero[i],
  PDstandard2nd[i_, j_] -> DPlus[i] DMinus[j],
  PDzero2nd[i_] -> DZero[i],
  PDzero2nd[i_, j_] -> DZero[i] DZero[j]
}
\end{verbatim}
\end{minipage}
\end{center}

In a calculation, a partial derivative is written in the form

\begin{center}
\begin{minipage}{0.8 \textwidth}
\begin{tt}
{\it name}[{\it gridfunction}, i, j, \ldots]
\end{tt}
\end{minipage}
\end{center}

For example, a one dimensional advection equation $\partial_t u =
\partial_x u$ with semidiscrete form $\partial_t v_j = D_{01} v_j$
could be described as

\begin{center}
\begin{minipage}{0.8 \textwidth}
\begin{tt}
dot[v] -> PDstandard2nd[v,1]
\end{tt}
\end{minipage}
\end{center}

The PartialDerivatives structure can also be used to define operators
for artificial dissipation.  Given a semidiscrete scheme
\begin{eqnarray}
\partial_t v(t)_j = F_j(v(t);t)
\end{eqnarray}
we can add Kreiss-Oliger style artificial dissipation by modifying the
scheme to read
\begin{eqnarray}
\partial_t v_j(t) = F_j(v(t);t) - \sigma \sum_i h_i^3 (D_{+i} D_{i})^2 v_j
\end{eqnarray}

We define a differencing operator {\tt Diss2nd} in the
PartialDerivatives structure with no directional arguments

\begin{center}
\begin{minipage}{0.8 \textwidth}
\begin{verbatim}
Diss2nd[] -> - sigma Sum[spacing[i]^3 (DPlus[i] DMinus[i])^2, 
                         {i, 1, 3}]
\end{verbatim}
\end{minipage}
\end{center}

using the standard Mathematica function for summations.  An evolution
equation representing the advection equation with dissipation could
then be written as

\begin{center}
\begin{minipage}{0.8 \textwidth}
\begin{verbatim}
dot[v] -> PDstandard2nd[v,1] + Diss2nd[v]
\end{verbatim}
\end{minipage}
\end{center}

A PartialDerivatives structure is given as an argument to the thorn
generation functions.

\subsection{Data structure: GroupDefinition}

A {\tt GroupDefinition} structure lists the grid functions that are
members of a specific Cactus group.  A list of such structures should
be supplied to CreateThorn function so that Kranc can determine which
group each grid function belongs to.

The form of a GroupDefinition structure is
\begin{center}
\begin{minipage}{0.8 \textwidth}
\begin{tt}
\{{\it groupname}, \{{\it variable1, variable2, \ldots}\}\}
\end{tt}
\end{minipage}
\end{center}
{\it groupname} is a string, and the {\it variables} are symbols.  For
example,
\begin{center}
\begin{minipage}{0.8 \textwidth}
\begin{tt}
\{"sol", \{phi, pi\}\}
\end{tt}
\end{minipage}
\end{center}
would represent a group called {\tt sol} with variables {\tt phi} and
{\tt pi}.

The group name can be prefixed with the name of the Cactus
implementation that provides the group followed by two colons
(e.g.~``ADMBase::metric'').

%%%%%%%%%%%%%%%%%%%%%%%%%%%%%%%%%%%%%%%%%%%%%%%%%%%%%%%%%%%%%%%%%%%%%%%%

\section{TensorTools}

The TensorTools package was written specifically for the Kranc system,
though it is in no way tied to it.  It is necessary to perform certain
operations on tensorial quantities, and there was no free software
available which met the requirements.

%% TensorTools has the following features:
%% \begin{itemize}
%% \item{It expands covariant derivatives in terms of partial derivatives
%% and Christoffel symbols (more than one covariant derivative can be
%% defined)}
%% \item{It expands Lie derivatives in terms of partial derivatives}
%% \item{Dummy indices can be automatically relabelled to avoid
%% conflicts}
%% \item{Abstract tensor expressions can be converted into component
%% expressions}
%% \end{itemize}

%% {\bf Note: I no longer use any of the above features apart from
%% expansion into components.  I prefer to do these type of calculation
%% by hand.  The reason is that the way the BSSN Gammas appear from the
%% Christoffel symbols is important, and they are not visible if you use
%% these techniques.}

\subsection{Representation of tensor quantities}

Tensorial expressions are entered in the same syntax as is used by
MathTensor, a commercial tensor manipulation package which can be used
instead of TensorTools.  An abstract tensor consists of a {\em kernel}
and an arbitrary number of abstract {\em indices}, each of which can
be {\em upper} or {\em lower}.  Abstract indices are alphabetical
characters (a-z, A-Z) prefixed with either an l or a u depending on
whether the index is considered to be lower or upper.  The tensor is
written using square brackets as
\begin{center}
\begin{tt}
kernel [ indices separated by commas ]
\end{tt}
\end{center}
%
For example, $T_a^{\phantom{a}b}$ would be written as {\tt T[la,ub]}.
There is no automatic index raising or lowering with any metric.
%
Entering a tensorial expression causes it to be displayed in standard
mathematical notation:
\mathdialogue{T[la,lb]}{$T_{ab}$}
%
Internally, tensors are represented as {\tt Tensor[{\it kernel},
TensorIndex[{\it label}, {\it type}], ...]} where {\it label} is the
alphabetical index, and {\it type} is either ``u'' or ``l'' depending
on the position of the index.  This representation helps in pattern
matching, and allows TensorTools to identify whether a certain object
is a tensor or not.

\subsection{Expansion of tensorial expressions into components}

As an example, the TensorTools function {\tt MakeExplicit} converts an
expression containing abstract tensors into a list of component
expressions:
\begin{center}
  \begin{tabular}[t]{rl}
  {\tt In := } & \parbox{10cm}{\tt MakeExplicit[T[la, lb]g[ub, uc]]} \\
\\
  {\tt Out = } &  \begin{tabular}[t]{rll}
\{ & g11 T11 & + g21 T12 + g31 T13, g12 T11 + g22 T12 + g32 T13, \\
   & g13 T11 & + g23 T12 + g33 T13, g11 T21 + g21 T22 + g31 T23, \\
   & g12 T21 & + g22 T22 + g32 T23, g13 T21 + g23 T22 + g33 T23, \\
   & g11 T31 & + g21 T32 + g31 T33, g12 T31 + g22 T32 + g32 T33, \\
   & g13 T31 & + g23 T32 + g33 T33\}\\
  \end{tabular} \\
  \end{tabular}
  \end{center}

Note here that there is no distinction made between upper and lower
indices in the component form.  TensorTools was written mainly for
automated code generation rather than symbolic manipulation; different
kernels should be used for the different forms if this is a problem.

%% \subsection{Covariant derivatives}

%% TensorTools allows the user to define more than one covariant
%% derivative.  The following defines a covariant derivative operator
%% {\tt CD} with Christoffel symbol {\tt H}:
%% %
%% \mathinput{DefineConnection[CD,H]}
%% %
%% The function {\tt CDtoPD} is used to replace covariant derivatives
%% with partial derivatives in any expression:
%% %
%% \mathdialogue
%% {
%%   CDtoPD[CD[V[ua],lb]]
%% }
%% {
%%   $V^a,\,_b + H^a_{\phantom{a}bc} V^c$
%% }
%% %
%% The function {\tt MakeExplicit} will automatically do this before
%% converting expressions into components.  In order to convert an
%% expression containing a covariant derivative into components,
%% TensorTools first simplifies the expression.  In the following, $x$
%% and $y$ represent expressions which may contain tensorial indices.  The
%% following steps are performed to simplify the expression:
%% \begin{itemize}
%% \item{Replace any high order covariant derivatives with repeated
%% application of a first order covariant derivative.  This ensures that
%% we only need to know how to evaluate a first derivative.
%% $$\nabla_d \nabla_a V^b \to \nabla_d ( \nabla_a V^b)$$}
%% \item{Replace the covariant derivative of a product using the Leibniz
%% rule: $$ \nabla_a (x y) \to (\nabla_a x) y + x (\nabla_a y) $$}
%% \item{Replace the covariant derivative of a sum using the linearity
%% property: $$ \nabla_a (x + y) \to \nabla_a x + \nabla_a y $$}
%% \item{Replace the covariant derivative of an arbitrary expression
%% containing tensorial indices with its expansion in terms of a
%% partial derivative and Christoffel symbols, one for each
%% index in the expression:  e.g.~$$\nabla_a V^b \to \partial_a V^b +
%% \Gamma^{b}_{\phantom{b}ac} V^c$$ }
%% \end{itemize}


%% \subsection{Lie derivatives}

%% The Lie derivative of an expression $x$ with respect to a vector $V$ is
%% written
%% \mathinput{Lie[x,V]}
%% where $V$ has been registered using {\tt DefineTensor} and is written {\em without}
%% indices.  The function {\tt LieToPD} is used to replace Lie derivatives
%% with partial derivatives:
%% \mathdialogue
%% {
%%   LieToPD[Lie[T[ua,lb], V]]
%% }
%% {
%%   $\Tud T a {b,c} V^c + \Tud T a c \Tud V c {,b} - \Tud T c b V^a_{,c}$
%% }
%% %
%% Lie derivatives of products and sums are supported.
%% %
%% The function {\tt MakeExplicit} will automatically perform this replacement
%% before converting expressions into components.


%% \subsection{Automatic dummy index manipulation}

%% When two expressions both containing a dummy index $b$ are multiplied
%% together, one dummy index is relabelled so as not to conflict with any
%% other index in the resulting expression:

%% \mathdialogue
%% {(T[la, lb]g[ub, uc])v[ub, ld, lb]}
%% {$T_{ab} g^{bc} V^e_{\phantom{e}de}$}
%% %
%% This requires that every multiplication be checked for tensorial
%% operands.  This can be a performance problem, so the feature can be
%% enabled and disabled with {\tt SetEnhancedTimes[True]} and {\tt
%% SetEnhancedTimes[False]}.  It is enabled by default.


\section{Creating a Kranc thorn}

Prototype: CreateThorn[groups, directory, thornName, namedArgs]\\

Note that if you want to use TensorTools tensors in calculations, you
must call the CreateThornTT function instead of this one.  It takes the
same arguments, but they can be tensorial in nature.

\subsubsection{Positional Arguments}

\begin{center}
\begin{tabularx}{\tablewidth}{|l|l|X|}
  \hline
  \bf Argument & \bf Type & \bf Description \\
  \hline
  groups & list of GroupDefinition structures & These define what groups each grid function is a member of.  \\
  directory & string & What directory to create the thorn in.  Usually ".".  \\ 
  thornName & string & The name to give the thorn.  \\
  namedArgs & Rule & The named arguments (see below) \\
  \hline
\end{tabularx}
\end{center}

\subsubsection{Named Arguments}

\begin{center}
\begin{tabularx}{\tablewidth}{|l|X|X|l|}
  \hline
  \bf Argument & \bf Type & \bf Description & \bf Default\\
  \hline
  Calculations & list of Calculation structures & The calculations to perform & \{\} \\

  DeclaredGroups & list of strings & The names of groups present in the {\tt groups} argument 
                                     which are to be created as new groups by this thorn.  & \{\} \\

  PartialDerivatives & a PartialDerivatives structure & The partial derivative definitions that
                                                        are used in this thorn (optional).  & \{\} \\

  RealParameters & list of strings & A list of real-valued parameters that this thorn will
                                     define and use.  They will all default to zero.  (optional) & \{\} \\

  IntParameters & list of strings & A list of integer-valued parameters that this thorn will
                                     define and use.  They will all default to zero.  (optional) & \{\} \\

  KeywordParameters & list of KeywordParameterDefinition structures & A list of parameter definition structures for all the keyword parameters which this thorn will define and use.  (optional) & \{\} \\

  InheritedImplementations & list of strings & A list of all the implementations which this thorn
                                               will inherit from.  This is necessary to use grid functions
                                               provided by these implementations.  (optional) & \{\} \\
  \hline
\end{tabularx}
\end{center}


%%%%%%%%%%%%%%%%%%%%%%%%%%%%%%%%%%%%%%%%%%%%%%%%%%%%%%%%%%%%%%%%%%%%%%%%






\chapter{Kranc internal design}

Kranc is composed of several Mathematica packages.  Each of these
human readable scripts performs a distinct function.
The diagram in Figure \ref{fig:kranc_design} illustrates the
relationships between the Kranc packages KrancThorns, TensorTools,
CodeGen, Thorn and MapLookup, which are described in the following
subsections.
\begin{figure}
\centering
\label{fig:kranc_design}
%\includegraphics[clip,width=0.9\textwidth]{KrancStructureLandscape.eps}
\caption{Relationships between Kranc packages: 
Each block represents a package, with the main functions it provides
indicated with square brackets.  An arrow indicates that one package
calls functions from another}
\end{figure}
Separating the different logically independent components of Kranc
into different packages promotes code reuse.  For example, none of the
thorn generation packages need to know anything about tensors, and
none of the packages other than CodeGen need to know the programming
language in which the thorn is being generated (C or Fortran).  We
have chosen to define several types of thorn (setter, evaluator, {\em
etc.}) but the mechanics of producing a thorn implemented in Thorn and
CodeGen are completely independent of this decision.

\subsection{Package: CodeGen}

During the development of the Kranc system, we explored two different
approaches to generating Cactus files using Mathematica as a
programming language.  Initially, a very straightforward system was
used whereby C statements were included almost verbatim in the
Mathematica script and output directly to the thorn source file.  This
approach has two main deficiencies:
\begin{itemize}
\item{The same block of text might be used in several places in the
code.  When a bug is fixed in one place, it must be fixed in all.}
\item{It is not easy to alter the language that is produced.  For
example, it is difficult to output both C and Fortran.}
\item{The syntax in the Mathematica source file is ugly, with lots of
string concatenation, making it difficult to read and edit}.
\end{itemize}

The CodeGen package provides functions to solve these problems.  To
address the first problem, Mathematica functions are used to represent
each block of code.  This allows the block to be customized by giving
the function arguments.  By making this abstraction, it became very
easy to change between outputting C and Fortran.

Fundamental to the system is the notion of a {\em block}; in
Mathematica terms this can be either a string or a list of blocks
(this definition is recursive).  All the CodeGen functions return
blocks, and the lists are all flattened and the strings concatenated
when the final source file is generated.  This is because it is
syntactically easier in the Mathematica source file to write a
sequence of statements as a list than to concatenate strings.

Many programming constructs are naturally block-structured; for
example, C {\tt for} loops need braces after the block of code to loop
over.  For this reason, it was decided that CodeGen functions could
take as arguments any blocks of code which needed to be inserted on
the inside of such a structure.

\subsection{Package: Thorn}

The Thorn package is used by all the different thorn generators to
construct the final Cactus thorn.  It takes care of the mechanics of
writing files to storage and parsing the Kranc structures necessary
for writing parameter configuration files, grid function definitions
etc.


\end{document}
